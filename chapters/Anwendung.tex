\chapter{Anwendung von B-B\"aume}

Heutzutage werden die B-B\"aume in Datenbanksystemen benutzt, die sehr hohe Datenmengen haben. Diese Datenstruktur verk\"urzt die Zeit die es ben\"otigt, um eine gew\"ahlte Menge von Daten zu suchen. Haupts\"achlich wird der B-Baum in Luft- und Raumfahrtindustrie eingesetzt.

\paragraph{}
Ein gutes Beispiel ist ein PC den man zu Hause stehen hat. Ein PC besitzt ein Hauptspeicher und eine Festplatte. Da der Hauptspeicher eine schnellen Lese/Schreiben-Zugriff haben muss, k\"onnte man unm\"oglich die ganzen Datenmengen in dem Hauptspeicher speichern. Denn dies wurde den Zugriff verlangsamen. Deshalb werden im Hauptspeicher lediglich die Werte der Schl\"ussel gespeichert und die gro\ss en Datenmengen werden auf der Festplatte gespeichert. Die Schl\"ussel haben die Verweise auf die Sektorbl\"ocke der Festplatte. Dies erh\"oht die Lese- und Schreibgeschwindigkeit, denn die Zugriff maximiert sich auf die Tiefe des Baumes.