\chapter{Einleitung}
\begin{flushleft}	
Die Aufgabe von Datenbanken ist die Verwaltung von gro�en Datenmengen. D.h. man
will verschiedene Operationen auf diesen Daten m�glichst schnell ausf�hren k�nnen.
Man muss die Informationen also in einer Datenstruktur abspeichern, die die
entsprechenden Operationen in einer m�glichst effizienten Weise unterst�tzen.
Man muss sich also die Frage stellen, welche Datenstruktur hierf�r geeignet ist und warum?
B-B�ume wurden von Prof. Rudolf Bayer explizit f�r die Plattenspeicherverwaltung
entworfen. F�r seine Arbeiten rund um den B-Baum und andere Verdienste erhielt Prof. 
Rudolf Bayer im Jahr (2001) den \glqq SIGMOD Innovations award\grqq. In unserer Arbeit setzen wir uns unter anderem mit dem Grund auseinander, warum B-B\"aume f�r die Plattenspeicherverwaltung geeignet ist. Dabei ist es zwangsweise notwendig die B-B\"aume genauer zu betrachten, und die in dieser Arbeit behandelten Themen n\"aher zu behandeln.
Zun�chst werden wir die B-B\"aume genauer definieren, gefolgt von der Behandlung der Operationen in B-B\"aumen. Anschlie\ss end werden noch die Komplexit\"atsklassen angesprochen und eine genauere Beschreibung der Anwendungen.
\end{flushleft}